\chapter{Related Work} 


Autonomous platooning is an active area of research, and several companies[] and research organizations[] are working on developing and testing platooning systems for a variety of vehicles, including trucks, buses, and passenger cars.

The development of autonomous platooning systems for heavy trucks has gained significant attention in recent years as a means of improving energy efficiency and reducing carbon emissions. In \cite{lit_1}, the authors focus on the development of a forward environment perception method for the fully automated lead vehicle of the autonomous platooning system. To achieve this, it is important to robustly detect obstacles from a distance and recognize which lane the obstacles exist in. The authors propose a multi-sensor fusion scheme using LIDAR and RADAR to detect obstacles robustly and precisely. Additionally, the authors use a digital map-based lane marker detection method to recognize the lanes where the obstacles exist. The proposed method is expected to enhance the safety and efficiency of autonomous platooning systems for heavy trucks.

In \cite{lit_2} the study highlights a significant limitation of existing platooning solutions, which are unable to support the formation of a 'long' platoon comprised of numerous vehicles, particularly long-body heavy-duty trucks, due to limited range in vehicle-to-vehicle communication protocols like DSRC and device-to-device communication for C-V2X. To overcome this challenge, we propose L-Platooning, the first platooning protocol that enables seamless, reliable, and rapid formation of a long platoon. L-Platooning introduces the concept of a Virtual Leader, which extends the coverage of the original platoon leader. We develop a virtual leader election algorithm based on a novel metric called the Virtual Leader Quality Index (VLQI) to effectively designate a virtual leader. Mechanisms for L-Platooning support vehicle join and leave maneuvers specific to a long platoon. Extensive simulations demonstrate that L-Platooning allows vehicles to effectively form a long platoon by accurately maintaining the desired inter-vehicle distance and handles vehicle join and leave maneuvers seamlessly.

Veichle platooning is also investigated by \cite{lit_3}. Vehicle platooning, which is a crucial step towards achieving fully automated driving, heavily relies on wireless communication for both platooning control input and environmental perception by platooning members (PMs). However, the limited availability of communication resources may result in excessive delay in platooning management, thus endangering platooning safety. To address this issue, this paper proposes an energy detection scheduling (EDS) scheme aimed at reducing platooning delay. Specifically, we analyze the impact of vehicle join and leave events on the original platoon's communication delay and propose using spectrum sensing to increase the number of communication resources. We then apply a local optimization algorithm to solve the resource allocation problem and introduce the EDS algorithm. The MATLAB simulation results show that the proposed EDS scheme significantly reduces platooning delay, surpassing traditional resource allocation schemes by more than half.


Autonomous Vehicle Platooning (AVP) has emerged as a promising solution to address the challenges of Intelligent Transportation Systems. However, effectively managing the join and leave requests of platoon vehicles while ensuring the platoon leader's profitability remains a challenge. This paper \cite{lit_4} proposes a dynamic AVP management protocol using the Ethereum platform. In this protocol, vehicles that want to join or leave the platoon communicate with the platoon leader via transactions that are regulated by smart contracts. To enhance cost-efficiency, a hybrid chain model is established, where certification records are stored on the public chain, and platoon communication records are kept on the privacy chain, which are later uploaded to the public chain as platoon operation incident records. The evaluation results and security analysis demonstrate the practicality and efficiency of the proposed scheme for AVP scenarios.


In \cite{lit_5} the study aims to investigate the fundamental maneuvers of platooning and propose detailed protocols for them. The protocols are evaluated on two types of platoons: homogeneous and heterogeneous. The study introduces the platoon configuration, maneuvering protocols, and the necessary control architecture for both longitudinal and lateral motion of vehicles. To assess the performance and safety of the most challenging platoon maneuvers - middle joining and leaving of a vehicle from the convoy - a visualization tool based on the Robot Operating System (ROS) is employed. The results indicate that all the discussed maneuvers lead to a stable platoon with the desired tracking performance and gap control. Overall, this study provides valuable insights for realizing the benefits of platooning in highways.