\chapter{Introduction}

The goal of this project is to develop an autonomous platooning system using a Husky Unmanned Ground Vehicle (UGV) and a TurtleBot3. The idea is to create a system where the TurtleBot3 can follow the Husky, while the Husky drives autonomously, creating a convoy or platoon of robots. This type of system has potential applications in various areas such as logistics, transportation, and search and rescue. 

To achieve this goal, the project will leverage the capabilities of ROS2, which is a popular open-source robotics framework that provides a flexible and modular architecture for developing robotic systems. ROS2 will be used for communication between the Husky and the TurtleBot3, allowing them to exchange data such as sensor measurements and control signals. This will enable the two robots to work together, with the Husky leading and the TurtleBot3 following behind.

To enable autonomous driving, the project will use the Navigation2 (NAV2) stack, which is a software framework for robot localisation, navigation and mapping. With NAV2, robots can achieve autonomous navigation by avoiding obstacles and staying on track.

This project consist of equipping TurtleBot3 and Husky with sensors and configure them to drive autonomously using ROS2 and NAV2. Examining different approaches to achieving autonomous platooning. The ones achieved will be tested and discussed.  

Overall, the development of an autonomous platooning system using a Husky UGV and a TurtleBot3 is an exciting and challenging project. Leveraging ROS2 and NAV2, the project will create a flexible and scalable system that can be adapted for various use cases and environments.

\section{Literature review}
Autonomous platooning is an active area of research, and several companies[] and research organizations[] are working on developing and testing platooning systems for a variety of vehicles, including trucks, buses, and passenger cars.

