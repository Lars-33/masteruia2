\chapter*{Sammendrag}

Platooning refererer til de koordinerte aktivitetene blant flere kjøretøy. I denne masteroppgaven utvides dette konseptet til autonom platooning. I denne sammenhengen navigerer et ubemannet kjøretøy autonomt (som leder), og i tillegg følger et annet kjøretøy (kalt følger) dens navigasjonsbane. To algoritmer (a) mimic og (b) time delay blir presentert. I tilfellet med førstnevnte imiterer følgeren lederens handling øyeblikkelig, mens i sistnevnte forsøker følgeren å følge lederens bane. Gjennom detaljert simulering og støttet av eksperimenter, demonstrerer denne oppgaven en autonom platooning av to ubemannede kjøretøy på laboratorieforhold. Anvendelsen av den brukte metoden for autonom platooning kan brukes i ulike applikasjonsområder som effektiv trafikkflyt og logistikk.