\chapter*{Abstract}

Platooning  refers to the co-ordinated activities among multiple vehicles. In this thesis, we extend this concept to the autonomous platooning. In this context, an unmanned ground vehicle navigates autonomously (acting as a leader), and in addition another vehicle (called follower) follows its navigational path. Two algorithms (a) mimic and (b) time delayed are presented. In the case of former the follower just imitates the action of leader instantaneously, while in the latter, follower tries to follow the trajectory of the leader. Through details simulation and backed by experiments, we demonstrate the autonomous platooning of two unmanned ground vehicle at laboratories conditions. The application of proposed method for autonomous platooning could be used in diverse application areas such as efficient traffic flow, and logistic.

Video abstract : \url{https://youtu.be/xEWZ0N4L8yI}

\iffalse
%aj write abstract
%aj mention about git
%aj attach link to video with results

%aj make a video abstract. 2-3 min long. cut the section of video where the mimic and time delay is playing.

Platooning  refers to the co-ordinated activities among multiple vehicles. In this thesis, we extend this concept to the autonomous platooning. In this context, an unmanned ground vehicle navigates autonomously (acting as a leader), and in addition another vehicle (called follower) follows its navigational path. Two algorithms (a) mimic and (b) time delayed are presented. In the case of former the follower just imitates the action of leader instantaneously, while in the latter, follower tries to follow the trajectory of the leader. Through details simulation and backed by experiments, we demonstrate the successful autonomous platooning of two unmanned ground vehicle at laboratories conditions. The application of proposed method for autonomous platooning could be used in diverse application areas such as efficient traffic flow, and logistic.
\fi